%%%%%%%%%%%%%%%%%%%%%%%%%%%%%%%%%%%%%%%%%
% Beamer Presentation
% LaTeX Template
% Version 1.0 (10/11/12)
%
% This template has been downloaded from:
% http://www.LaTeXTemplates.com
%
% License:
% CC BY-NC-SA 3.0 (http://creativecommons.org/licenses/by-nc-sa/3.0/)
%
%%%%%%%%%%%%%%%%%%%%%%%%%%%%%%%%%%%%%%%%%

%----------------------------------------------------------------------------------------
%	PACKAGES AND THEMES
%----------------------------------------------------------------------------------------

\documentclass{beamer}

\mode<presentation> {

% The Beamer class comes with a number of default slide themes
% which change the colors and layouts of slides. Below this is a list
% of all the themes, uncomment each in turn to see what they look like.

%\usetheme{default}
%\usetheme{AnnArbor}
%\usetheme{Antibes}
%\usetheme{Bergen}
%\usetheme{Berkeley}
%\usetheme{Berlin}
%\usetheme{Boadilla}
%\usetheme{CambridgeUS}
%\usetheme{Copenhagen}
%\usetheme{Darmstadt}
%\usetheme{Dresden}
%\usetheme{Frankfurt}
%\usetheme{Goettingen}
%\usetheme{Hannover}
%\usetheme{Ilmenau}
%\usetheme{JuanLesPins}
%\usetheme{Luebeck}
\usetheme{Madrid}
%\usetheme{Malmoe}
%\usetheme{Marburg}
%\usetheme{Montpellier}
%\usetheme{PaloAlto}
%\usetheme{Pittsburgh}
%\usetheme{Rochester}
%\usetheme{Singapore}
%\usetheme{Szeged}
%\usetheme{Warsaw}

% As well as themes, the Beamer class has a number of color themes
% for any slide theme. Uncomment each of these in turn to see how it
% changes the colors of your current slide theme.

%\usecolortheme{albatross}
%\usecolortheme{beaver}
%\usecolortheme{beetle}
%\usecolortheme{crane}
%\usecolortheme{dolphin}
%\usecolortheme{dove}
%\usecolortheme{fly}
%\usecolortheme{lily}
%\usecolortheme{orchid}
%\usecolortheme{rose}
%\usecolortheme{seagull}
%\usecolortheme{seahorse}
%\usecolortheme{whale}
%\usecolortheme{wolverine}

%\setbeamertemplate{footline} % To remove the footer line in all slides uncomment this line
%\setbeamertemplate{footline}[page number] % To replace the footer line in all slides with a simple slide count uncomment this line

%\setbeamertemplate{navigation symbols}{} % To remove the navigation symbols from the bottom of all slides uncomment this line
}

\usepackage{graphicx} % Allows including images
\usepackage{booktabs} % Allows the use of \toprule, \midrule and \bottomrule in tables
\usepackage{url,latexsym,amsmath,amsthm,xspace,rotating,multirow,multicol,xspace,amssymb,paralist}


\newcommand{\obs}{\text{obs}}
\newcommand{\mis}{\text{mis}}

\newcommand{\qt}[1]{\left<#1\right>}
\newcommand{\ql}[1]{\left[#1\right]}
\newcommand{\hess}{\mathbf{H}}
\newcommand{\jacob}{\mathbf{J}}
\newcommand{\hl}{HL}
\newcommand{\cost}{\mathcal{L}}
\newcommand{\lout}{\mathbf{r}}
\newcommand{\louti}{r}
\newcommand{\outi}{y}
\newcommand{\out}{\mathbf{y}}
\newcommand{\gauss}{\mathbf{G_N}}
\newcommand{\eye}{\mathbf{I}}
\newcommand{\softmax}{\text{softmax}}
\newcommand{\targ}{\mathbf{t}}
\newcommand{\metric}{\mathbf{G}}
\newcommand{\sample}{\mathbf{z}}
\newcommand{\f}{\text{f}}
%\newcommand{\log}{\text{log}}

\newcommand{\bmx}[0]{\begin{bmatrix}}
\newcommand{\emx}[0]{\end{bmatrix}}
\newcommand{\qexp}[1]{\left<#1\right>}
\newcommand{\vect}[1]{\mathbf{#1}}
\newcommand{\vects}[1]{\boldsymbol{#1}}
\newcommand{\matr}[1]{\mathbf{#1}}
\newcommand{\var}[0]{\operatorname{Var}}
\newcommand{\std}[0]{\operatorname{std}}
\newcommand{\cov}[0]{\operatorname{Cov}}
\newcommand{\diag}[0]{\operatorname{diag}}
\newcommand{\matrs}[1]{\boldsymbol{#1}}
\newcommand{\va}[0]{\vect{a}}
\newcommand{\vb}[0]{\vect{b}}
\newcommand{\vc}[0]{\vect{c}}
\newcommand{\ve}[0]{\vect{e}}

\newcommand{\vh}[0]{\vect{h}}
\newcommand{\vv}[0]{\vect{v}}
\newcommand{\vx}[0]{\vect{x}}
\newcommand{\vp}[0]{\vect{p}}
\newcommand{\vz}[0]{\vect{z}}
\newcommand{\vw}[0]{\vect{w}}
\newcommand{\vs}[0]{\vect{s}}
\newcommand{\vf}[0]{\vect{f}}
\newcommand{\vi}[0]{\vect{i}}
\newcommand{\vo}[0]{\vect{o}}
\newcommand{\vd}[0]{\vect{d}}
\newcommand{\vy}[0]{\vect{y}}
\newcommand{\vg}[0]{\vect{g}}
\newcommand{\vm}[0]{\vect{m}}
\newcommand{\vu}[0]{\vect{u}}
\newcommand{\vL}[0]{\vect{L}}
\newcommand{\vr}[0]{\vect{r}}
\newcommand{\vone}[0]{\vect{1}}

\newcommand{\mW}[0]{\matr{W}}
\newcommand{\mE}[0]{\matr{E}}
\newcommand{\mG}[0]{\matr{G}}
\newcommand{\mX}[0]{\matr{X}}
\newcommand{\mY}[0]{\matr{Y}}
\newcommand{\mQ}[0]{\matr{Q}}
\newcommand{\mU}[0]{\matr{U}}
\newcommand{\mF}[0]{\matr{F}}
\newcommand{\mV}[0]{\matr{V}}
\newcommand{\mA}{\matr{A}}
\newcommand{\mC}{\matr{C}}
\newcommand{\mD}{\matr{D}}
\newcommand{\mL}[0]{\matr{L}}
\newcommand{\mR}[0]{\matr{R}}
\newcommand{\mS}{\matr{S}}
\newcommand{\mI}{\matr{I}}
\newcommand{\td}[0]{\text{d}}
\newcommand{\TT}[0]{\vects{\theta}}
\newcommand{\vsig}[0]{\vects{\sigma}}
\newcommand{\valpha}[0]{\vects{\alpha}}
\newcommand{\vmu}[0]{\vects{\mu}}
\newcommand{\vzero}[0]{\vect{0}}
\newcommand{\tf}[0]{\text{m}}
\newcommand{\tdf}[0]{\text{dm}}
\newcommand{\grad}[0]{\nabla}
\newcommand{\N}[0]{\mathcal{N}}
\newcommand{\YY}[0]{\mathcal{Y}}
\newcommand{\BB}[0]{\mathcal{B}}
\newcommand{\LL}[0]{\mathcal{L}}
\newcommand{\HH}[0]{\mathcal{H}}
\newcommand{\RR}[0]{\mathbb{R}}
\newcommand{\MM}[0]{\mathcal{M}}
\newcommand{\OO}[0]{\mathbb{O}}
\newcommand{\II}[0]{\mathbb{I}}
\newcommand{\Scal}[0]{\mathcal{S}}
\newcommand{\sigmoid}{\sigma}
\newcommand{\E}[0]{\mathbb{E}}
\newcommand{\enabla}[0]{\ensuremath{%
    \overset{\raisebox{-0.3ex}[0.5ex][0ex]{%
    \ensuremath{\scriptscriptstyle e}}}{\nabla}}}
\newcommand{\enhnabla}[0]{\nabla_{\hspace{-0.5mm}e}\,}
\newcommand{\eos}[0]{\ensuremath{\left< \text{eos}\right>}}


\newcommand{\todo}[1]{{\Large\textcolor{red}{#1}}}
\newcommand{\done}[1]{{\Large\textcolor{green}{#1}}}
\newcommand{\dd}[1]{\ensuremath{\mbox{d}#1}}

\DeclareMathOperator*{\argmax}{\arg \max}
\DeclareMathOperator*{\argmin}{\arg \min}
\newcommand{\newln}{\\&\quad\quad{}}

\newcommand{\BP}{\text{BP}}
\newcommand{\PPL}{\text{PPL}}
\newcommand{\PL}{\text{PL}}
\newcommand{\MatSum}{\text{MatSum}}
\newcommand{\MatMul}{\text{MatMul}}
\newcommand{\KL}{\text{KL}}
\newcommand{\data}{\text{data}}
\newcommand{\rect}{\text{rect}}
\newcommand{\maxout}{\text{maxout}}
\newcommand{\train}{\text{train}}
\newcommand{\val}{\text{val}}
\newcommand{\init}{\text{init}}
\newcommand{\fenc}{\text{fenc}}
\newcommand{\renc}{\text{renc}}
\newcommand{\enc}{\text{enc}}
\newcommand{\dec}{\text{dec}}
\newcommand{\test}{\text{test}}
\newcommand{\Ax}{\mathcal{A}_x}
\newcommand{\Ay}{\mathcal{A}_y}
\newcommand{\ola}{\overleftarrow}
\newcommand{\ora}{\overrightarrow}
\newcommand{\ov}{\overline}
\newcommand{\ts}{\rule{0pt}{2.6ex}}       % Top strut
\newcommand{\ms}{\rule{0pt}{0ex}}         % Middle strut
\newcommand{\bs}{\rule[-1.2ex]{0pt}{0pt}} % Bottom strut
\newcommand{\specialcell}[2][c]{%
  \begin{tabular}[#1]{@{}c@{}}#2\end{tabular}}

\renewcommand{\alert}[1]{\textcolor{red}{#1}}

%----------------------------------------------------------------------------------------
%	TITLE PAGE
%----------------------------------------------------------------------------------------

\title[Week 3: MLPs and Backprop]{Week 3: MLPs and Backprop} % The short title appears at the bottom of every slide, the full title is only on the title page

\author{NYU DS-GA 3001} % Your name
\institute[] % Your institution as it will appear on the bottom of every slide, may be shorthand to save space
{
Slides by Sam Bowman\\
Based on lecture notes by Kyunghyun Cho
 \\ % Your institution for the title page
\medskip
\textit{bowman@nyu.edu} % Your email address
}
\date{9/20/16} % Date, can be changed to a custom date

\begin{document}

\begin{frame}
\titlepage % Print the title page as the first slide
\end{frame}

\begin{frame}
\frametitle{Overview} % Table of contents slide, comment this block out to remove it
\tableofcontents % Throughout your presentation, if you choose to use \section{} and \subsection{} commands, these will automatically be printed on this slide as an overview of your presentation
\end{frame}

%----------------------------------------------------------------------------------------
%	PRESENTATION SLIDES
%----------------------------------------------------------------------------------------

\section{Introductions \& Administrativia}

\begin{frame}
\frametitle{Introductions \& Administrativia}

\begin{itemize}
\item You
% Who has taken a class that uses DL?
% CDS students
% CS masters
% PhD
% UG
% Say your name before questions!
\item Me (hi again!)
\item Changes from the original syllabus. So far:
\begin{itemize}
\item MT (pending)
\item Topics
\item Slides
\end{itemize}
\item Questions?
\end{itemize}
\end{frame}

\begin{frame}
\frametitle{Overview} 
\tableofcontents
\end{frame}

\section{Multilayer Perceptron}

\begin{frame}
\frametitle{Multilayer Perceptron}

\begin{itemize}
\item The \textit{classic} deep artificial neural network (ca. 1962).
\item Structure:
\begin{itemize}
\item Input $\vx$
\item ...feeds a sequence of one or more hidden layers
\begin{align*}
    \phi_0(\vx) = g(\mW_0 \vx + \vb_0),
\end{align*}
Where $g\in$
\begin{itemize}
        \item Sigmoid: $\sigma(x) = \frac{1}{1+\exp(-x)}$
        \item Hyperbolic function: $\tanh(x) = \frac{1-\exp(-2x)}{1+\exp(-2x)}$
        \item Rectified linear unit: $\rect(x) = \max(0, x)$
    \end{itemize}
\item ...which feed an output layer (usually softmax)
\end{itemize}
\end{itemize}

\end{frame}

\subsection{Example: Binary classification with a single hidden unit}


\begin{frame}
\frametitle{Simplest possible MLP: Structure}

\begin{itemize}
\item Structure:
\begin{itemize}
\item Scalar input $x \in \RR$
\item ...feeds one hidden layer
\begin{align*}
    \phi(x) = \sigma(u x + c)
\end{align*}
\item ...which feeds a sigmoid output layer representing a (Bernoulli) conditional probability $p(y|x)$ over a Boolean output variable $y$.
\begin{align*}
    \mu = f(x) = \sigma(w \phi(x) + b)
\end{align*}
\end{itemize}
\item ...where
\begin{align*}
    \sigma(x) = \frac{1}{1+\exp(-x)}.
\end{align*}
\item Four scalar parameters: $u, c, w, b$
\end{itemize}

\end{frame}


\begin{frame}
\frametitle{Simplest possible MLP: Cost Function}

\begin{itemize}
\item Objective: Minimize KL divergence between
\begin{itemize}
\item true (empirical) conditional distribution $p(y|x)$ 
\item predicted conditional distribution $\hat{p}(y|x)$
\end{itemize}
\begin{align*}
    \KL(p\|\hat{p}) =& \sum_{y \in \left\{0, 1\right\}} p(y|x) \log
    \frac{p(y|x)}{\hat{p}(y|x)} \\
    =& \sum_{y \in \left\{0, 1\right\}} p(y|x) \log p(y|x) - p(y|x) \log
    {\hat{p}(y|x)}.
\end{align*}
\item Equivalent to minimizing:
\begin{align*}
 - \sum_{y \in \left\{0, 1\right\}} p(y|x) \log {\hat{p}(y|x)}.
\end{align*}
\end{itemize}

\end{frame}



\begin{frame}
\frametitle{Simplest possible MLP: Cost Function}

\begin{itemize}
\item When we compute gradients for gradient descent, we'll be computing the \textit{per sample} version of this cost function and averaging
\begin{align*}
    C_x =& - \log \hat{p}(y|x) \\
    =& - \log \mu^y (1-\mu)^{1-y} \\
    =& - y \log \mu - (1-y) \log (1-\mu)
\end{align*}
\item Our task: Define the gradient for each of our four parameters wrt. $C_x$: $\frac{\partial C_x}{\partial u}, \frac{\partial C_x}{\partial c}, \frac{\partial C_x}{\partial w}, \frac{\partial C_x}{\partial b}$

\end{itemize}

\end{frame}




\begin{frame}
\frametitle{Simplest possible MLP: Gradient}

\begin{itemize}
\item Let's start with $w$ and use the chain rule:
\begin{align*}
    \frac{\partial C_x}{\partial w} = \frac{\partial C_x}{\partial \mu}
    \frac{\partial \mu}{\partial \underline{\mu}}
    \frac{\partial \underline{\mu}}{\partial w}
\end{align*}
($\underline{\mu} = w \phi(x) + b$, which is the input to $f$)
\item Starting from the left...
\begin{align*}
    \frac{\partial C_x}{\partial \mu} \underbrace{\frac{\partial \mu}{\partial
    \underline{\mu}}}_{=\mu'} & = 
    -\frac{y}{\mu}\mu' +
    \frac{1-y}{1-\mu}\mu'\\ 
   & = \frac{-y + y\mu + \mu - y\mu }{\mu(1-\mu)}\mu' 
    = \frac{\mu - y}{\mu(1-\mu)}\mu' 
    = \mu - y
\end{align*}
(the derivative of $\sigma(\mu)$ is $\mu (1-\mu)$.)
\end{itemize}

\end{frame}

\begin{frame}
\frametitle{Simplest possible MLP: Gradient}

\begin{itemize}
\item Continuing with $\frac{\partial C_x}{\partial w}$, all we need is: 
\begin{align*}
\frac{\partial \underline{\mu}}{\partial w} = \phi(x)
\end{align*}
\item So:
\begin{align*}
\frac{\partial C_x}{\partial w} = \frac{\partial C_x}{\partial \mu}
    \frac{\partial \mu}{\partial \underline{\mu}}
    \frac{\partial \underline{\mu}}{\partial w} = (\mu - y)\phi(x)
\end{align*}
\end{itemize}

\end{frame}


\begin{frame}
\frametitle{Simplest possible MLP: Gradient}

\begin{itemize}
\item Now let's try $b$ and use the chain rule:
\begin{align*}
    \frac{\partial C_x}{\partial b} = \frac{\partial C_x}{\partial \mu}
    \frac{\partial \mu}{\partial \underline{\mu}}\color{black}
    \frac{\partial \underline{\mu}}{\partial b}
\end{align*}
\item Starting from the left, we need...
\begin{align*}
    \frac{\partial C_x}{\partial \mu} \underbrace{\frac{\partial \mu}{\partial
    \underline{\mu}}}_{=\mu'} & = 
    -\frac{y}{\mu}\mu' +
    \frac{1-y}{1-\mu}\mu'\\ 
   & = \frac{-y + y\mu + \mu - y\mu }{\mu(1-\mu)}\mu' 
    = \frac{\mu - y}{\mu(1-\mu)}\mu' 
    = \mu - y
\end{align*}
\end{itemize}

\end{frame}

\begin{frame}
\frametitle{Simplest possible MLP: Gradient}


\begin{itemize}
\item Now let's try $b$ and use the chain rule:
\begin{align*}
    \frac{\partial C_x}{\partial b} = \color{green}\frac{\partial C_x}{\partial \mu}
    \frac{\partial \mu}{\partial \underline{\mu}}\color{black}
    \frac{\partial \underline{\mu}}{\partial b}
\end{align*}
\item Starting from the left, we need...
\begin{align*}
    \frac{\partial C_x}{\partial \mu} \underbrace{\frac{\partial \mu}{\partial
    \underline{\mu}}}_{=\mu'} & = 
    -\frac{y}{\mu}\mu' +
    \frac{1-y}{1-\mu}\mu'\\ 
   & = \frac{-y + y\mu + \mu - y\mu }{\mu(1-\mu)}\mu' 
    = \frac{\mu - y}{\mu(1-\mu)}\mu' 
    = \mu - y
\end{align*}
\item But we already computed that! [Foreshadowing.]
\end{itemize}

\end{frame}

\begin{frame}
\frametitle{Simplest possible MLP: Gradient}

\begin{itemize}
\item On to $u$ and $c$. Let's use the chain rule again...
\begin{align*}
    \frac{\partial C_x}{\partial u} =& \color{green} \frac{\partial C_x}{\partial
    \underline{\mu}}
    \color{blue} \frac{\partial \underline{\mu}}{\partial \phi}
    \frac{\partial \phi}{\partial \underline{\phi}}
    \color{black} \frac{\partial \underline{\phi}}{\partial u} \\
    \frac{\partial C_x}{\partial c} =& \color{green}\frac{\partial C_x}{\partial
    \underline{\mu}}
    \color{blue} \frac{\partial \underline{\mu}}{\partial \phi}
     \frac{\partial \phi}{\partial \underline{\phi}}
    \color{black} \frac{\partial \underline{\phi}}{\partial c}
\end{align*}
\end{itemize}

\end{frame}

\begin{frame}
\frametitle{Simplest possible MLP: Gradient}

\begin{itemize}
\item Continuing with what we still need...
\begin{align*}
    &\frac{\partial \underline{\mu}}{\partial \phi}\underbrace{\frac{\partial \phi}{\partial
    \underline{\phi}}}_{=\phi'} = w \phi' = w \phi(x) (1- \phi(x))\\
    &\frac{\partial \underline{\phi}}{\partial u} = x \\
    &\frac{\partial \underline{\phi}}{\partial c} = 1
\end{align*}
\item And we're done!
\begin{align*}
    \frac{\partial C_x}{\partial u} =& (\mu -y) w \phi(x) (1-\phi(x)) x \\
    \frac{\partial C_x}{\partial c} =& (\mu -y) w \phi(x) (1-\phi(x))
\end{align*}
\end{itemize}

\end{frame}


\subsection{Example: Binary classification with more than one hidden unit}


\begin{frame}
\frametitle{A more realistic MLP: Structure}

\begin{itemize}
\item Structure:
	\begin{itemize}
	\item Vector input $x \in \RR^d$
	\item ...feeds one vector-valued hidden layer
	\begin{align*}
	    \phi(x) = \sigma(U x + c)
	\end{align*}
	where $U \in \RR^{l \times d}$ and $c \in \RR^l$
	\item ...which feeds a sigmoid output layer representing a (Bernoulli) conditional probability $p(y|x)$ over a Boolean output variable $y$.
	\begin{align*}
	    \mu = f(x) = \sigma(w \phi(x) +  b)
	\end{align*}
	where $w \in \RR^{l}$ and $b \in \RR$
	\end{itemize}
\item Four parameters: $U, c, w, b$
\item In all cases we'll be looking at, the nonlinearity (here $\sigma$) is applied elementwise.
\end{itemize}

\end{frame}

\begin{frame}
\frametitle{A more realistic MLP: Gradient}

\begin{itemize}
\item Keeping the cost as before, let's start with $W$ and use the chain rule:
\begin{align*}
    \frac{\partial C_x}{\partial w} = \color{green}\frac{\partial C_x}{\partial \mu}
    \frac{\partial \mu}{\partial \underline{\mu}}
    \color{black}\frac{\partial \underline{\mu}}{\partial w}
\end{align*}
\item $w$ is now a vector, so the function computing $\underline{\mu}$ is now more complex
\begin{align*}
    \underline{\mu} = w^\top \phi(x) + b = \sum_{i=1}^l w_i \phi_i(x) + b.
\end{align*}
\item Let's start by thinking of $\frac{\partial \underline{\mu}}{\partial w}$ as a vector of scalar partial derivatives 
\begin{align*}
    \frac{\partial \underline{\mu}}{\partial w} = \left[ 
        \frac{\partial \underline{\mu}}{\partial w_1}, \frac{\partial
        \underline{\mu}}{\partial w_2}, \ldots, \frac{\partial
        \underline{\mu}}{\partial w_l}
    \right]^\top = \left[ \phi_1(x), \phi_2(x), \ldots, \phi_l(x) \right]^\top
    = \phi(x)
\end{align*}
\end{itemize}

\end{frame}


\begin{frame}
\frametitle{A more realistic MLP: Gradient}

\begin{itemize}
\item So, as before,
\begin{align*}
    \frac{\partial C_x}{\partial w} & = \frac{\partial C_x}{\partial \mu}
    \frac{\partial \mu}{\partial \underline{\mu}}
    \color{black}\frac{\partial \underline{\mu}}{\partial w}\\
& = (\mu -y) \phi(x)
\end{align*}
\item Similarly, 
\begin{align*}
    \frac{\partial C_x}{\partial b} & = \frac{\partial C_x}{\partial \mu}
    \frac{\partial \mu}{\partial \underline{\mu}}
    \color{black}\frac{\partial \underline{\mu}}{\partial b}\\
& = \mu -y
\end{align*}
\end{itemize}
\end{frame}


\begin{frame}
\frametitle{A more realistic MLP: Gradient}

\begin{itemize}
\item Now back to $u$ and $c$. Let's use the chain rule again...
\begin{align*}
    \frac{\partial C_x}{\partial U} =& \color{green} \frac{\partial C_x}{\partial
    \underline{\mu}}
    \color{blue} \frac{\partial \underline{\mu}}{\partial \phi}
    \frac{\partial \phi}{\partial \underline{\phi}}
    \color{black} \frac{\partial \underline{\phi}}{\partial U} \\
    \frac{\partial C_x}{\partial c} =& \color{green}\frac{\partial C_x}{\partial
    \underline{\mu}}
    \color{blue} \frac{\partial \underline{\mu}}{\partial \phi}
     \frac{\partial \phi}{\partial \underline{\phi}}
    \color{black} \frac{\partial \underline{\phi}}{\partial c}
\end{align*}
\item We can use the same vector trick as just before (taking advantage of some symmetry) to show
\begin{align*}
    \frac{\partial \underline{\mu}}{\partial \phi} = w
\end{align*}
\end{itemize}

\end{frame}


\begin{frame}
\frametitle{A more realistic MLP: Gradient}


\begin{itemize}
\item Because $\sigma$ is elementwise, we can simply
compute its derivative for each element in $\phi(x)$
\begin{align*}
    \frac{\partial \phi}{\partial \underline{\phi}} = 
    \diag\left(\left[ \phi_1'(x), \phi_2'(x), \ldots, \phi_l'(x)
    \right]^\top\right)
\end{align*}
where $\diag$ returns a diagonal matrix of the input vector
\item This gets us 
\begin{align*}
\frac{\partial C_x}{\partial \underline{\phi}} = 
    (\mu - y)w^\top \phi'(x) = (\mu - y) (w \odot \diag(\phi'(x))),
\end{align*}
\end{itemize}

\end{frame}

\begin{frame}

\frametitle{A more realistic MLP: Gradient}


\begin{itemize}
\item On to the last step...
\begin{align*}
    \frac{\partial \underline{\phi}}{\partial U} = \frac{\partial U^\top
    x}{\partial U} = x
\end{align*}
\item So...
\begin{align} 
    \frac{\partial C_x}{\partial U} = (\mu - y) (w \odot \diag(\phi'(x)))
    x^\top
\end{align}
\item And we can easily generalize to $c$:
\begin{align} 
    \frac{\partial C_x}{\partial U} = (\mu - y) (w \odot \diag(\phi'(x)))
\end{align}
\item Now all we have to do is hope SGD converges! Yay?
\end{itemize}

\end{frame}

\section{Automating Backpropagation}

\begin{frame}
\frametitle{Visualizing Backpropagation}

\begin{figure}[ht]
    \centering
    \begin{minipage}{0.48\textwidth}
        \centering
        \includegraphics[width=0.9\textwidth]{../figures/comp_graph.pdf}
    \end{minipage}
    \hfill
    \begin{minipage}{0.48\textwidth}
        \centering
        \includegraphics[width=0.9\textwidth]{../figures/comp_node.pdf}
    \end{minipage}

    \begin{minipage}{0.48\textwidth}
        \centering
        (a)
    \end{minipage}
    \hfill
    \begin{minipage}{0.48\textwidth}
        \centering
        (b)
    \end{minipage}
    \caption{(a) A graphical representation of the computational graph of the
    example network. (b) A graphical illustration
    of a function node ({\color{blue} $\to$}: forward pass, {\color{red}
$\leftarrow$}: backward pass.)}
\end{figure}

\end{frame}


\begin{frame}
\frametitle{Visualizing Backpropagation}

\begin{figure}[ht]
    \centering
    \begin{minipage}{0.48\textwidth}
        \centering
        \includegraphics[width=0.9\textwidth]{../figures/comp_graph.pdf}
    \end{minipage}
    \hfill
    \begin{minipage}{0.48\textwidth}
        \begin{itemize}
        \item If you can compute the local gradients for the function nodes, backprop is easy (and easy to automate).
        \item Inventory of common function nodes is pretty small (dozens), easy to find good library code w/ gradients.
        \item So, use TensorFlow/Theano/Torch/etc. and never apply the chain rule yourself again!
        \end{itemize}
    \end{minipage}
\end{figure}

\end{frame}


\begin{frame}
\frametitle{Summing up: Backpropagation}

\begin{itemize}
\item Nothing more than clever application of the chain rule
\item If we compute the gradients we need in the right order (moving backwards through the \textit{computation graph}), each one will:
\begin{itemize}
\item Require only simple functions of local variables \textit{and/or} partial gradients we've already computed
\item Therefore take only about as long as the corresponding computation in the forward pass
\end{itemize}
\item So gradient computation takes about as long as forward computation!
\end{itemize}

\end{frame}



\begin{frame}
\frametitle{Bonus issue: Gradients as graphs}

\begin{itemize}
\item Why not represent gradient computation as a second computation graph derived from the first?
\item Lets us compute gradients of (gradients of (gradients of)) gradients, visualize gradients with the same tools we use for the rest of the graph, etc.
\item Implemented in Torch, Theano
\end{itemize}

\end{frame}

\subsection{What if a Function is {\em not} Differentiable?}

\begin{frame}
\frametitle{Differentiability}

\begin{itemize}
\item Backpropagation requires that every function node is differentiable wrt. its inputs and parameters
\item All the standard NN building blocks are:
\begin{itemize}
\item Matrix multiplication (normal or elementwise), addition: See above and/or Matrix Cookbook
\item Sigmoid nonlinearity:  $\sigma'(x) = \sigma(x) (1 - \sigma(x))$
\item $\tanh$ nonlinearity:  $\tanh'(x) = \left(\frac{2}{\exp(x)+\exp(-x)}\right)^2$
\end{itemize}
\end{itemize}

\end{frame}

\begin{frame}
\frametitle{Differentiability}

\begin{itemize}
\item ReLU is a bit more complicated:
\begin{align*}
\rect'(x) = \left\{ 
        \begin{array}{l l}
            1, & \text{if }x > 0 \\
            0, & \text{if }x < 0
        \end{array}
        \right.
\end{align*}
\item Exact zero values for $x$ are rare enough to not matter (can stipulatively change $<$ to $\leq$)
\item Exact zero values for $\rect{x}$ are common, and block gradients from flowing backwards...
\item ...but only for some elements in some examples, so it learns anyway
\end{itemize}

\end{frame}

\begin{frame}
\frametitle{Differentiability}

\begin{itemize}
\item What if you want to incorporate a function that just isn't differentiable?
\item This includes most functions you might dream up! Examples:
\begin{itemize}
\item Sampling from a multinomial distribution $P(m|x)$
\item Looking up the $int(x)$-th row of a matrix
\item Identifying the argmax index into $x$
\end{itemize}
\item You can compute gradients and use SGD, but you'll generally have to resort to slower and less reliable techniques (mostly: reinforcement learning). Stay tuned 'til the end of the course.
\end{itemize}

\end{frame}

\section{Next time}

\begin{frame}
\begin{itemize}
\item Lab tomorrow
\item Reading: 
\item Lecture: RNNs and word vectors!
\item HW 1: Due in eight days
\end{itemize}
\end{frame}


%----------------------------------------------------------------------------------------

\end{document} 
